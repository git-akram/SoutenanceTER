\documentclass{beamer} 


%======================================================================
% Packages / Options
%======================================================================

\usepackage[utf8]{inputenc}

\usepackage[T1]{fontenc}
\usepackage{listings}
\lstset{language=caml,basicstyle=\ttfamily}

% Beamer-specific settings
\mode<presentation>
{
  \usetheme{Darmstadt}
  \useoutertheme{infolines}
}

\AtBeginSection[]
{
  \begin{frame}<beamer>
    \frametitle{Plan}
    \tableofcontents[sectionstyle=show/shaded,subsectionstyle=hide]
  \end{frame}
}

\AtBeginSubsection[]
{
  \begin{frame}<beamer>
    \frametitle{Plan}
    \tableofcontents[sectionstyle=show/hide,subsectionstyle=show/shaded/hide]
  \end{frame}
}

\setbeamertemplate{footline}
{%
\insertpagenumber
\insertshorttitle[width={3cm},center]
\insertshortinstitute[width={3cm},center]
\insertshortdate[width={3cm},center]
}


%======================================================================
% Titlepage
%======================================================================

\title[Analyse Syntaxique]{Analyse Syntaxique}

\author{Mohammed Akram RHAFRANE - Mehdi BOUTCHICHE - Nathanael BERTRAND - Ismail SENHAJI}
\institute{Université de Toulouse III/IRIT}
\date{Année 2012/2013} 

\begin{document}

\begin{frame}
  \titlepage
\end{frame}


%======================================================================
\section{Objectifs du TER}\label{sec:Objectifs}

Initiation à la recherche bibliographique.

Rédaction :
\idem Rapport sur le sujet (40 pages).
\idem Rapport de stage (10 pages).
\idem Présentation du TER.



\frame

% Section 1---------------------------------------------------
\section{Fondamentaux}

\subsection{Langage et grammaire}



\subsection{Analyse Lexicale}



\subsection{Analyse Syntaxique}



\subsubsection{Analyse LL}



\subsubsection{Analyse LR}



% Section 2---------------------------------------------------
\section{Comparaison entre les outils}

\subsection{YACC}



\subsubsection{Définition}



\subsubsection{Exemple}



\subsection{ANTLR}



\subsubsection{Définition}



\subsubsection{Exemple}



% Section 3---------------------------------------------------
\section{Xtext}

\subsection{Définition}



\subsection{Fonctionnement}



\subsection{Exemple}



%-------------------------------------------------------------

\section{Conclusion}

\end{document}

%%% Local Variables: 
%%% mode: latex
%%% TeX-master: t
%%% coding: utf-8
%%% End: 
